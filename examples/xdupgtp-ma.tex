\documentclass[ma]{xdupgtp}

% 学号
\newcommand\studentNo{1101110071}
% 论文题目
\newcommand\thesisTitleOne{西安电子科技大学学位论文}
\newcommand\thesisTitleTwo{开题报告表XeLaTeX模板}
% 姓名
\newcommand\authorName{张三}
% 一级学科
\newcommand\major{电子科学与技术}
% 二级学科
\newcommand\submajor{电磁场与微波技术}
% 指导教师
\newcommand\advisorName{李四}
% 学院
\newcommand\school{电子工程学院}
% 开题日期
\newcommand\submitDate{2021年12月31日}

\begin{document}

\section{论文概况}
\begin{xtly}
% 在这里撰写选题来源
\end{xtly}
\begin{zwzy}
% 在这里撰写中文摘要
\end{zwzy}

\section{选题依据}
\begin{xtyj}
\subsection{选题意义}
% 在这里撰写选题意义
\subsection{国内外研究现状}
% 在这里撰写国内外研究现状
\end{xtyj}

\section{研究方案}
\begin{yjfa}
\subsection{研究目标}
% 在这里撰写研究目标
\subsection{研究内容}
% 在这里撰写研究内容
本部分内容要体现出学位论文的整体设想及构架。
\subsection{拟解决的关键问题}
% 在这里撰写拟解决的关键问题
\subsection{拟采取的研究方法、技术路线、实验方案及可行性研究}
% 在这里撰写拟采取的研究方法、技术路线、实验方案及可行性研究
\subsection{研究计划及预期取得的研究成果}
研究计划要具体,要明确指出每一个时间段的学位论文进展情况及预期取得的研究成果。
% 在这里撰写研究计划及预期取得的研究成果
\end{yjfa}

\section{研究基础}
\begin{yjjc}
\subsection{已具备的实验条件和研究工作积累}
% 在这里撰写已具备的实验条件和研究工作积累
\subsection{已取得的科研成果}
% 在这里撰写已取得的科研成果
\end{yjjc}

\section{指导教师意见}
\begin{jsyj}
% 在这里撰写指导教师意见
\end{jsyj}

\section{开题报告记录}
\begin{bgjl}
% 在这里撰写开题报告记录
\end{bgjl}

\section{开题报告评语及结论}
\begin{pyjl}
% 在这里撰写开题报告评语及结论
\end{pyjl}

\end{document}
