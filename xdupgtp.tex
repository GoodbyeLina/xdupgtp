\documentclass{xdupgtp}

% 学号
\newcommand\studentNo{1101110071}
% 论文题目
\newcommand\thesisTitleOne{西安电子科技大学学位论文}
\newcommand\thesisTitleTwo{开题报告表XeLaTeX模板}
% 姓名
\newcommand\authorName{张三}
% 一级学科/学位类别
\newcommand\major{电子科学与技术}
% 二级学科/领域(方向)
\newcommand\submajor{电磁场与微波技术}
% 指导教师 专业学位硕士研究生为校内导师
\newcommand\advisorName{李四}
% 学院
\newcommand\school{电子工程学院}
% 开题日期
\newcommand\submitDate{2021年12月31日}
% 选题来源
\newcommand\subjectSource{国家自然科学基金}
% 以下内容仅限专业学位硕士研究生填写
% 校外导师
\newcommand\entadvisorName{王五}
% 实习单位名称
\newcommand\practiceInst{西安电子科技大学}
% 实习岗位
\newcommand\practicePost{实习岗位}
% 实习实践模式 以下内容选择一种
% 校内
% 校外
\newcommand\practiceMode{校内}
% 计划实习时间
\newcommand\practiceTime{计划实习时间}
% 论文类型 以下内容选择一种
% 调研报告
% 工程(规划)设计
% 应用基础技术
% 实用新型技术
% 应用软件技术
% 技术报告
% 工程(项目)管理和案例分析
% 技术论文
\newcommand\thesisType{技术报告}
% 签名图像
% 组长签名
% \renewcommand\zzqm{\sign{sign1}}
% 成员签名
% \renewcommand\cyqm{\sign{sign2}\sign{sign3}\emptysign\sign{sign4}}
% 指导教师签名 学术学位研究生
% \renewcommand\zdjsqm{\sign{sign1}}
% 校内导师签名 专业学位研究生
% \renewcommand\xnjsqm{\sign{sign2}}
% 校外导师签名 专业学位研究生
% \renewcommand\xyjsqm{\sign{sign3}}

\begin{document}

\section{论文概况}
\begin{zwzy}
% 在这里撰写中文摘要
\end{zwzy}

\section{选题依据}
\begin{xtyj}
\subsection{选题意义}
% 在这里撰写选题意义
\subsection{国内外研究现状}
% 在这里撰写国内外研究现状
\end{xtyj}

\section{研究方案}
\begin{yjfa}
\subsection{研究目标}
% 在这里撰写研究目标
\subsection{研究内容}
% 在这里撰写研究内容
本部分内容要体现出学位论文的整体设想及构架。
\subsection{拟解决的关键问题}
% 在这里撰写拟解决的关键问题
\subsection{拟采取的研究方法、技术路线、实验方案及可行性研究}
% 在这里撰写拟采取的研究方法、技术路线、实验方案及可行性研究
\subsection{论文的创新点} % 硕士研究生请移除论文的创新点
% 博士研究生在这里撰写论文的创新点
\subsection{研究计划及预期取得的研究成果}
研究计划要具体,要明确指出每一个时间段的学位论文进展情况及预期取得的研究成果。
% 在这里撰写研究计划及预期取得的研究成果
\end{yjfa}

\section{研究基础}
\begin{yjjc}
\subsection{已具备的实验条件和研究工作积累}
% 在这里撰写已具备的实验条件和研究工作积累
\subsection{已取得的科研成果}
% 在这里撰写已取得的科研成果
\end{yjjc}

% 硕士研究生将指导教师意见移动至此处

\section{开题报告记录}
\begin{bgjl}
% 在这里撰写开题报告记录
\end{bgjl}

\section{开题报告评语及结论}
\begin{pyjl}
% 在这里撰写开题报告评语及结论
\end{pyjl}

\section{指导教师意见}
\begin{jsyj}
% 学术学位研究生
% 在这里撰写指导教师意见
\end{jsyj}
\begin{xnjsyj}
% 专业学位研究生
% 在这里撰写校内指导教师意见
\end{xnjsyj}
\begin{xyjsyj}
% 专业学位研究生
% 在这里撰写校外指导教师意见
\end{xyjsyj}

\end{document}
